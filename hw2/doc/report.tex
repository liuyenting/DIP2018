\documentclass[12pt]{article}

\usepackage[margin=1in]{geometry}
\usepackage{fancyhdr}
\pagestyle{fancy}
\usepackage{amsmath}
\usepackage{amssymb}

% limit to particular location
\usepackage{float}

% graphics
\usepackage{graphicx}
% for subfigures
\usepackage{subcaption}

% better ref links
\usepackage{hyperref}

% footnote in footer
\newcommand{\fancyfootnotetext}[2]{%
  \fancypagestyle{dingens}{%
    \fancyfoot[LO,RE]{\parbox{7cm}{\footnotemark[#1]\footnotesize #2}}%
  }%
  \thispagestyle{dingens}%
}

\lhead{HW2}
\chead{Digital Image Processing}
\rhead{B03902036}


\begin{document}


\section*{Problem 1}
\subsection*{First-order edge detection}
\paragraph*{Robert filter} calculates diagonal edge gradients (\autoref{eq:robert_op}), susceptible to fluctuations and provides no information about edge orientation.

\begin{equation}
\label{eq:robert_op}
\begin{cases}
	G_x = \begin{bmatrix}
		1 & 0 \\ 0 & -1 
	\end{bmatrix} \ast I \\
	G_y = \begin{bmatrix}
		0 & 1 \\ -1 & 0
	\end{bmatrix} \ast I
\end{cases}
\end{equation}

\noindent
Gradient map is composed of
\begin{equation}
\label{eq:gx_gy_gxy}
	\nabla I(x, y) = G(x, y) = \sqrt{G_x^2 + G_y^2}
\end{equation}
and the edge map is determined by
\begin{equation}
\label{eq:edge_map}
E = 
\begin{cases}
	1, 2 \text{ S.D.}(\sim 5\%) \\
	0, \text{otherwise}
\end{cases}
\end{equation}

\begin{figure}[H]
    \centering
    \begin{subfigure}[t]{0.5\textwidth}
        \centering
        \includegraphics[height=2.5in]{images/robert_x}
        \caption{$I_x$}
    \end{subfigure}%
    ~ 
    \begin{subfigure}[t]{0.5\textwidth}
        \centering
        \includegraphics[height=2.5in]{images/robert_y}
        \caption{$I_y$}
    \end{subfigure}
    ~
    \vskip\baselineskip
    \begin{subfigure}[t]{0.5\textwidth}
        \centering
        \includegraphics[height=2.5in]{images/robert_xy}
        \caption{$I_{xy}$}
    \end{subfigure}%
    ~
    \begin{subfigure}[t]{0.5\textwidth}
        \centering
        \includegraphics[height=2.5in]{images/robert_edge}
        \caption{Edge map}
    \end{subfigure}
    \caption{Robert operator}
\end{figure}

\paragraph*{Prewitt filter} is a discrete differentiation operator, which behaves similar to the Sobel operator by computing the gradient for the image intensity function as well.
It makes use of the maximum directional gradient. Though it is easy to implement, it is very sensitive to noise as well.

\begin{equation}
\label{eq:robert_op}
\begin{cases}
	G_x = \begin{bmatrix}
		-1 & 0 & 1 \\ -1 & 0 & 1 \\ -1 & 0 & 1 
	\end{bmatrix} \ast I\\
	G_y = \begin{bmatrix}
		-1 & -1 & -1 \\ 0 & 0 & 0 \\ 1 & 1 & 1
	\end{bmatrix} \ast I
\end{cases}
\end{equation}
The final gradient result calculates through \autoref{eq:gx_gy_gxy} and \autoref{eq:edge_map} as well.

\begin{figure}[H]
    \centering
    \begin{subfigure}[t]{0.5\textwidth}
        \centering
        \includegraphics[height=2.5in]{images/prewitt_x}
        \caption{$I_x$}
    \end{subfigure}%
    ~ 
    \begin{subfigure}[t]{0.5\textwidth}
        \centering
        \includegraphics[height=2.5in]{images/prewitt_y}
        \caption{$I_y$}
    \end{subfigure}
    ~
    \vskip\baselineskip
    \begin{subfigure}[t]{0.5\textwidth}
        \centering
        \includegraphics[height=2.5in]{images/prewitt_xy}
        \caption{$I_{xy}$}
    \end{subfigure}%
    ~
    \begin{subfigure}[t]{0.5\textwidth}
        \centering
        \includegraphics[height=2.5in]{images/prewitt_edge}
        \caption{Edge map}
    \end{subfigure}
    \caption{Prewitt operator}
\end{figure}

Noted that since Prewitt kernel can be decomposed, $G_x$ can be written as 
\begin{equation}
	G_x = \begin{bmatrix}
		-1 & 0 & 1 \\ -1 & 0 & 1 \\ -1 & 0 & 1 
	\end{bmatrix} \ast I
	= \begin{bmatrix}
		1 \\ 1 \\ 1 
	\end{bmatrix} \begin{bmatrix}
 		-1 & 0 & 1	
	\end{bmatrix}
	\ast I
\end{equation}
which essentially means that it computes the gradient with smoothing, since it can be decomposed as the products of an averaging and a differential kernel.

\paragraph*{Sobel filter} detects edges where the gradient magnitude is high. 
This makes the Sobel edge detector more sensitive to diagonal edge than horizontal (\autoref{fig:sobel_x}) and vertical edges (\autoref{fig:sobel_y}). 
Both Sobel filter and Prewitt filter are very effective at providing good edge maps.

\begin{equation}
\begin{cases}
	G_x = \begin{bmatrix}
		1 & 0 & -1 \\ 2 & 0 & -2 \\ 1 & 0 & -1 
	\end{bmatrix} \ast I \\
	G_y = \begin{bmatrix}
		1 & 2 & 1 \\ 0 & 0 & 0 \\ -1 & -2 & -1
	\end{bmatrix} \ast I
\end{cases}
\end{equation}

\noindent
The final gradient result calculates through \autoref{eq:gx_gy_gxy} and \autoref{eq:edge_map} as well.


\begin{figure}[H]
    \centering
    \begin{subfigure}[t]{0.5\textwidth}
        \centering
        \includegraphics[height=2.5in]{images/sobel_x}
        \caption{$I_x$}
        \label{fig:sobel_x}
    \end{subfigure}%
    ~ 
    \begin{subfigure}[t]{0.5\textwidth}
        \centering
        \includegraphics[height=2.5in]{images/sobel_y}
        \caption{$I_y$}
        \label{fig:sobel_y}
    \end{subfigure}
    ~
    \vskip\baselineskip
    \begin{subfigure}[t]{0.5\textwidth}
        \centering
        \includegraphics[height=2.5in]{images/sobel_xy}
        \caption{$I_{xy}$}
    \end{subfigure}%
    ~
    \begin{subfigure}[t]{0.5\textwidth}
        \centering
        \includegraphics[height=2.5in]{images/sobel_edge}
        \caption{Edge map}
    \end{subfigure}
    \caption{Sobel operator}
\end{figure}

\subsection*{Second-order edge detection}
If there is a significant spatial change in the second-derivative, an edge is detected. Second-order derivative operators are more sophisticated methods, yet, they are also very noise-sensitive. 

\paragraph*{Laplacian of Gaussian}
Laplacian filters are derivative filters used to find areas of rapid changing edges in images.

\begin{equation}
	\nabla^2 f(x, y) = \frac{\partial^2 f(x, y)}{\partial x^2} + \frac{\partial^2 f(x, y)}{\partial y^2}
\end{equation}

Since derivative filters are very sensitive to noise, it is common to smooth the image using a Gaussian filter before applying the Laplacian.

A possible approximation for the effect of the Laplacian is 
\begin{equation}
	K_L = \begin{bmatrix}
		0 & 1 & 0 \\ 1 & -4 & 1 \\ 0 & 1 & 0
	\end{bmatrix}
\end{equation}
One can reverse the sign of the elements of this negative Laplacian, but it does not affect the outcome.

To include a smoothing Gaussian lowpass filter, combine the Laplacian and Gaussian functions to obtain the result.
\begin{equation}
	K_{LoG} = -\frac{1}{\pi \sigma^4} \lbrack 1-\frac{x^2+y^2}{2 \sigma^2} \rbrack e^{-\frac{x^2 + y^2}{2 \sigma^2}}
\end{equation}
To utilize functions from previous homework, the calculation is still split instead of using a single kernel.

\begin{figure}[H]
    \centering
    \begin{subfigure}[t]{0.5\textwidth}
        \centering
        \includegraphics[height=2.5in]{images/log_raw}
        \caption{$I_1$}
    \end{subfigure}%
    ~ 
    \begin{subfigure}[t]{0.5\textwidth}
        \centering
        \includegraphics[height=2.5in]{images/log_g}
        \caption{$LPF(I_1)$}
        \label{fig:log_g}
    \end{subfigure}
    ~
    \vskip\baselineskip
    \begin{subfigure}[t]{0.5\textwidth}
        \centering
        \includegraphics[height=2.5in]{images/log_log}
        \caption{$LoG(I_1)$}
        \label{fig:log_log}
    \end{subfigure}%
    ~
    \begin{subfigure}[t]{0.5\textwidth}
        \centering
        \includegraphics[height=2.5in]{images/log_edge}
        \caption{Edge map}
    \end{subfigure}
    \caption{LoG}
\end{figure}
Both \autoref{fig:log_g} and \autoref{fig:log_log} uses kernel size of 5 and $\sigma=1$. \autoref{fig:log_log} is ranged from $\lbrack -1, 1 \rbrack$. 

\paragraph*{Difference of Gaussians} is a feature enhancement method that involves the subtraction of one blurred version of an original image from another less blurred version of the original. 
In other word, first-order derivative of the Gaussian blurred image.

\begin{figure}[H]
    \centering
    \begin{subfigure}[t]{0.3\textwidth}
        \centering
        \includegraphics[height=2in]{images/dog_s2}
        \caption{$K_1$, $\sigma=2$}
    \end{subfigure}%
    ~ 
    \begin{subfigure}[t]{0.3\textwidth}
        \centering
        \includegraphics[height=2in]{images/dog_s5}
        \caption{$K_2$, $\sigma=5$}
    \end{subfigure}%
    ~
    \begin{subfigure}[t]{0.3\textwidth}
        \centering
        \includegraphics[height=2in]{images/dog_kernel}
        \caption{$K = K_2-K_1$}
    \end{subfigure}
    \caption{LoG}
    \label{fig:dog_kernel}
\end{figure}

\autoref{fig:dog_kernel} is the kernel used in this section, which is the result of subtracting Gaussian kernel of $\sigma = 5$ and $\sigma=2$.

\begin{figure}[H]
    \centering
    \begin{subfigure}[t]{0.5\textwidth}
        \centering
        \includegraphics[height=2.5in]{images/dog_grad}
        \caption{DoG}
    \end{subfigure}%
    ~
    \begin{subfigure}[t]{0.5\textwidth}
        \centering
        \includegraphics[height=2.5in]{images/dog_edge}
        \caption{Edge map}
    \end{subfigure}
    \caption{LoG}
    \label{fig:dog}
\end{figure}

\subsection*{Canny edge detection}
\paragraph*{Step 1: Noise reduction}
A simple Gaussian lowpass filter is suffice for this task. $\sigma$ is chose to be 5.
\begin{figure}[H]
    \centering
    \includegraphics[height=2.5in]{images/canny_1}
    \caption{Noise reduction}
\end{figure}

\paragraph*{Step 2: Compute gradient magnitude and orientation}
The Sobel operator used in previous section
\begin{equation}
\begin{cases}
	G_x = \begin{bmatrix}
		1 & 0 & -1 \\ 2 & 0 & -2 \\ 1 & 0 & -1 
	\end{bmatrix} \ast I \\
	G_y = \begin{bmatrix}
		1 & 2 & 1 \\ 0 & 0 & 0 \\ -1 & -2 & -1
	\end{bmatrix} \ast I
\end{cases}
\end{equation}
was applied here as well, with gradients calculated as
\begin{equation}
	\nabla I(x, y) = G(x, y) = \sqrt{G_x^2 + G_y^2}
\end{equation}
and
\begin{equation}
	\Theta(x, y) = tan^{-1}\frac{G_y}{G_x}
\end{equation}
$\Theta$ was wrapped around from $[-180, 180]$ degrees to $[0, 360]$ to help through the mind blowing if-else clauses (please see the script for details).

\begin{figure}[H]
    \centering
    \begin{subfigure}[t]{0.5\textwidth}
        \centering
        \includegraphics[height=2.5in]{images/canny_sxy}
        \caption{Magnitude}
    \end{subfigure}%
    ~
    \begin{subfigure}[t]{0.5\textwidth}
        \centering
        \includegraphics[height=2.5in]{images/canny_sth}
        \caption{Orientation, $[0, 360]$ degrees}
    \end{subfigure}
    \caption{Sobel operator}
\end{figure}

\paragraph*{Step 3: Non-maximal suppression} searches its nearest neighbor along the edge normal.
\begin{equation}
	I(x, y) = \begin{cases}
		I(x, y), \text{for } I(x, y) > I(x^\prime, y^\prime) \text{ and } I(x, y) > I(x^{\prime\prime}, y^{\prime\prime})\\
		0, \text{otherwise}
	\end{cases}
\end{equation}
where $I(x^\prime, y^\prime)$ and $I(x^{\prime\prime}, y^{\prime\prime})$ are pre-determined by sorting the gradient directions to 0, 45, 90, or 135 degrees (\autoref{fig:canny_ori}).

\begin{figure}[H]
    \centering
    \begin{subfigure}[t]{0.5\textwidth}
        \centering
        \includegraphics[height=2.5in]{images/canny_ori}
        \caption{Sorted orientation map}
        \label{fig:canny_ori}
    \end{subfigure}%
    ~
    \begin{subfigure}[t]{0.5\textwidth}
        \centering
        \includegraphics[height=2.5in]{images/canny_3}
        \caption{Suppression result}
    \end{subfigure}
    \caption{Non-maximal suppression}
\end{figure}

\paragraph*{Step 4: Hysteresis thresholding} 
follows the instructions, 
\begin{equation}
I(x, y) = 
\begin{cases}
1, I(x, y) > T_H \\
0, I(x, y) < T_L \\
1, \text{8-connected}
\end{cases}
\end{equation}
where $T_H$ and $T_L$ are the upper and lower strict boundary. Pixels that falls in between is determined by its 8-connected neighbor for its true nature.
\begin{figure}[H]
    \centering
    \includegraphics[height=2.5in]{images/canny_4}
    \caption{Hysteresis thresholding, $T_H = 0.5$, $T_L = 0.4$}
\end{figure}
This is the final result.

\subsection*{Edges of noisy image}
Since the noise is periodic, by analyzing the frequency domain, we can extract the components (\autoref{fig:freq_domain_masked}) through auto-correlation of the frequency domain and find the peaks through donut-shaped masking. Diameter of the peak mask is 5 pixels.
\begin{figure}[H]
    \centering
    \begin{subfigure}[t]{0.5\textwidth}
        \centering
        \includegraphics[height=2.5in]{images/log_abs_I2}
        \caption{$log(\vert FT(I_2) \vert)$}
    \end{subfigure}%
    ~
    \begin{subfigure}[t]{0.5\textwidth}
        \centering
        \includegraphics[height=2.5in]{images/mask_X}
        \caption{Components masked}
        \label{fig:freq_domain_masked}
    \end{subfigure}
    \caption{Frequency domain}
\end{figure}
\noindent
This leads to
\begin{figure}[H]
    \centering
    \begin{subfigure}[t]{0.5\textwidth}
        \centering
        \includegraphics[height=2.5in]{images/stripe_before}
        \caption{Before filtering ($I_2$)}
    \end{subfigure}%
    ~
    \begin{subfigure}[t]{0.5\textwidth}
        \centering
        \includegraphics[height=2.5in]{images/stripe_after}
        \caption{After filtering}
    \end{subfigure}
    \caption{Filtering}
    \label{fig:stripe_filter}
\end{figure}
\noindent
Though we can further masking the harmonic frequency components, for this homework, this should be enough to continue.

\begin{figure}[H]
    \centering
    \begin{subfigure}[t]{0.5\textwidth}
        \centering
        \includegraphics[height=2.5in]{images/stripe_before_edge}
        \caption{Before filtering}
    \end{subfigure}%
    ~
    \begin{subfigure}[t]{0.5\textwidth}
        \centering
        \includegraphics[height=2.5in]{images/stripe_after_edge}
        \caption{After filtering}
    \end{subfigure}
    \caption{Edge map comparison}
    \label{fig:stripe_edges}
\end{figure}

To exaggerate the effect of filtering, I choose to use Robert operator with lower-then-normal threshold -- cut at average instead of above 2 S.D. -- palm region in \autoref{fig:stripe_edges} can clearly see that the filtering has indeed done its job.

\section*{Problem 2}
\subsection*{Edge crisping}
Basic operation is based on the unsharp filter
\begin{equation}
	C(x, y) = I_3(x, y) - K_n(\sigma) \ast I_3(x, y)
\end{equation}

\noindent
while $n$ is the kernel size and $\sigma$ is the standard deviation of the Gaussian function.
\autoref{fig:crisp_demo} demonstrates the generic effect of an unsharp filter.
\begin{figure}[H]
    \centering
    \begin{subfigure}[t]{0.3\textwidth}
        \centering
        \includegraphics[height=2in]{images/crisp_raw}
        \caption{$I_3$}
    \end{subfigure}%
    ~ 
    \begin{subfigure}[t]{0.3\textwidth}
        \centering
        \includegraphics[height=2in]{images/crisp_smooth}
        \caption{$K \ast I_3$}
    \end{subfigure}%
    ~
    \begin{subfigure}[t]{0.3\textwidth}
        \centering
        \includegraphics[height=2in]{images/crisp_crisp}
        \caption{$C$}
    \end{subfigure}
    \caption{LoG}
    \label{fig:crisp_demo}
\end{figure}
where $n$ is 15 and $\sigma$ is 5. One can easily distinguish the edge of the Milky Way and high visibility stars.

\begin{figure}[H]
    \centering
    \begin{subfigure}[t]{0.3\textwidth}
        \centering
        \includegraphics[height=2in]{images/crisp_n5s5}
        \caption{$n=5, \sigma=5$}
    \end{subfigure}%
    ~ 
    \begin{subfigure}[t]{0.3\textwidth}
        \centering
        \includegraphics[height=2in]{images/crisp_n15s5}
        \caption{$n=15, \sigma=5$}
    \end{subfigure}%
    ~
    \begin{subfigure}[t]{0.3\textwidth}
        \centering
        \includegraphics[height=2in]{images/crisp_n25s5}
        \caption{$n=25, \sigma=5$}
    \end{subfigure}
    ~
    \vskip\baselineskip
    \begin{subfigure}[t]{0.3\textwidth}
        \centering
        \includegraphics[height=2in]{images/crisp_n15s1}
        \caption{$n=15, \sigma=1$}
    \end{subfigure}%
    ~ 
    \begin{subfigure}[t]{0.3\textwidth}
        \centering
        \includegraphics[height=2in]{images/crisp_n15s10}
        \caption{$n=15, \sigma=10$}
    \end{subfigure}%
    ~
    \begin{subfigure}[t]{0.3\textwidth}
        \centering
        \includegraphics[height=2in]{images/crisp_n15s100}
        \caption{$n=15, \sigma=100$}
    \end{subfigure}%
    \caption{LoG}
    \label{fig:crisp_tune}
\end{figure}

While both increasing the kernel size and $\sigma$ enhance the result, increasing the kernel size while locking the $\sigma$ (first row in \autoref{fig:crisp_tune}) keeps the impulse noises indicates by the stars, which is reasonable due to the fact that out-skirt of the subtracted kernel is mostly zero, which is effectively a smaller blur kernel.

The second row in \autoref{fig:crisp_tune} increase the $\sigma$ while keep the kernel size at $n=5$, gradually increase the weighting of the surrounding pixels, hence the signals of stars gradually diminished (lower left region). 

When taking into consideration greater variety of the edge differences, aka, increase $n$, edge difference is more profound, due to considering less localized variations.
Increasing the $\sigma$ increases the differences per variation in pixel intensity, hence the features has greater contrast when comparing results between $n=15$ and $\sigma=5, 10$.

\subsection*{Warping}
Using reverse lookup of the following equations
\begin{equation}
\begin{cases}
	x(u, v) = u + 20 sin(\frac{2 \pi}{2} \frac{y}{512} ) \\
	y(u, v) = v + 20 sin(\frac{2 \pi}{3} \frac{x}{512} )
\end{cases}
\end{equation}

\begin{figure}[H]
    \centering
    \begin{subfigure}[t]{0.5\textwidth}
        \centering
        \includegraphics[height=2.5in]{images/warp_mask}
        \caption{Mask}
    \end{subfigure}%
    ~
    \begin{subfigure}[t]{0.5\textwidth}
        \centering
        \includegraphics[height=2.5in]{images/warp_c}
        \caption{D}
    \end{subfigure}
    \caption{Warping with $sin$ functions}
    \label{fig:warp}
\end{figure}
Though Prof. Lee told us something interesting will show up, \autoref{fig:warp} is, somehow, quite disappointing :(

\section*{Bonus}
I implement a simple rolling ball algorithm using combination of erosion then dilation through a ball-shaped structure element.
\autoref{fig:sample4} shows that relative intensity between features are maintained, while non-uniform background illumination is roughly removed as requested by Prof. Lee. 

\begin{figure}[H]
    \centering
    \begin{subfigure}[t]{0.5\textwidth}
        \centering
        \includegraphics[height=2.5in]{images/roll_sample4}
        \caption{Before}
    \end{subfigure}%
    ~
    \begin{subfigure}[t]{0.5\textwidth}
        \centering
        \includegraphics[height=2.5in]{images/roll_bkgsub}
        \caption{After}
    \end{subfigure}
    \caption{Sample 4}
    \label{fig:sample4}
\end{figure}

Ball diameter is chosen to be 60 pixels instead of 15, determined by the size of features in the sample image - larger components roughly have diameter of 60, while smaller one is 15 - though using larger diameter may risk removing the smaller features. From \autoref{fig:background_diff}, they would not matter much, only the larger components have dramatic intensity change when switching size. \autoref{fig:background_60} indicates that background is removed more thoroughly (aggressively).

\begin{figure}[H]
    \centering
    \begin{subfigure}[t]{0.3\textwidth}
        \centering
        \includegraphics[height=2in]{images/roll_blo}
        \caption{15 pixels}
        \label{fig:background_15}
    \end{subfigure}%
    ~
    \begin{subfigure}[t]{0.3\textwidth}
        \centering
        \includegraphics[height=2in]{images/roll_bhi}
        \caption{60 pixels}
        \label{fig:background_60}
    \end{subfigure}%
    ~
    \begin{subfigure}[t]{0.3\textwidth}
        \centering
        \includegraphics[height=2in]{images/roll_bd}
        \caption{(b)-(a)}
        \label{fig:background_diff}
    \end{subfigure}
    \caption{Background isolated by different ball size}
    \label{fig:background}
\end{figure}
\noindent
Background determined by different ball diameter is illustrated in \autoref{fig:background}.
Smaller components are visible in \autoref{fig:background_15} as described in previous paragraph.

If contrast-limited adaptive histogram equalization (CLAHE) is applied, the result is even better. CAUTION, CLAHE IS APPLIED THROUGH IMAGEJ INSTEAD OF USING MATLAB, CLAHE DEMO IS ONLY FOR PROOF-OF-CONCEPT. Block size is 65, number of bins per block is 256, slope limitation is 3.0.
\begin{figure}[H]
    \centering
    \begin{subfigure}[t]{0.5\textwidth}
        \centering
        \includegraphics[height=2.5in]{images/roll_bkgsub}
        \caption{Rolling ball}
    \end{subfigure}%
    ~
    \begin{subfigure}[t]{0.5\textwidth}
        \centering
        \includegraphics[height=2.5in]{images/roll_clahe}
        \caption{CLAHE}
    \end{subfigure}
    \caption{CLAHE for sample 4}
    \label{fig:sample4_clahe}
\end{figure}

Another sample image,
\begin{figure}[H]
    \centering
    \begin{subfigure}[t]{0.3\textwidth}
        \centering
        \includegraphics[height=2in]{images/roll_sample5}
        \caption{Original}
    \end{subfigure}%
    ~
    \begin{subfigure}[t]{0.3\textwidth}
        \centering
        \includegraphics[height=2in]{images/roll_sample5_hi}
        \caption{Rolling ball}
    \end{subfigure}%
    ~
    \begin{subfigure}[t]{0.3\textwidth}
        \centering
        \includegraphics[height=2in]{images/roll_sample5_clahe}
        \caption{CLAHE}
        \label{fig:sample5_clahe}
    \end{subfigure}
    \caption{Background isolated by different ball size}
    \label{fig:sample5}
\end{figure}
Though CLAHE can limit exaggerated noise in constant intensity region, \autoref{fig:sample5_clahe} shows that slope of 2 is not enough for noise suppression, yet, if we further decrease the slope limitation, CLAHE degrade into generic adaptive local histogram equalization.

\end{document}
              