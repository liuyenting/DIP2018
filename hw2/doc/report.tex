\documentclass[12pt]{article}

\usepackage[margin=1in]{geometry}
\usepackage{fancyhdr}
\pagestyle{fancy}
\usepackage{amsmath}
\usepackage{amssymb}

% limit to particular location
\usepackage{float}

% graphics
\usepackage{graphicx}
% for subfigures
\usepackage{subcaption}

% better ref links
\usepackage{hyperref}

% footnote in footer
\newcommand{\fancyfootnotetext}[2]{%
  \fancypagestyle{dingens}{%
    \fancyfoot[LO,RE]{\parbox{7cm}{\footnotemark[#1]\footnotesize #2}}%
  }%
  \thispagestyle{dingens}%
}

\lhead{HW2}
\chead{Digital Image Processing}
\rhead{B03902036}


\begin{document}


\section*{Problem 1}
\subsection*{First-order edge detection}
\paragraph*{Robert filter} calculates diagonal edge gradients (\autoref{eq:robert_op}), susceptible to fluctuations and provides no information about edge orientation.

\begin{equation}
\label{eq:robert_op}
\begin{cases}
	G_x = \begin{bmatrix}
		1 & 0 \\ 0 & -1 
	\end{bmatrix} \ast I \\
	G_y = \begin{bmatrix}
		0 & 1 \\ -1 & 0
	\end{bmatrix} \ast I
\end{cases}
\end{equation}

\noindent
Gradient map is composed of
\begin{equation}
\label{eq:gx_gy_gxy}
	\nabla I(x, y) = G(x, y) = \sqrt{G_x^2 + G_y^2}
\end{equation}
and the edge map is determined by
\begin{equation}
\label{eq:edge_map}
E = 
\begin{cases}
	1, 2 \text{ S.D.}(\sim 5\%) \\
	0, \text{otherwise}
\end{cases}
\end{equation}

\begin{figure*}[ht!]
    \centering
    \begin{subfigure}[t]{0.5\textwidth}
        \centering
        \includegraphics[height=3in]{images/robert_x}
        \caption{$I_x$}
    \end{subfigure}%
    ~ 
    \begin{subfigure}[t]{0.5\textwidth}
        \centering
        \includegraphics[height=3in]{images/robert_y}
        \caption{$I_y$}
    \end{subfigure}
    ~
    \vskip\baselineskip
    \begin{subfigure}[t]{0.5\textwidth}
        \centering
        \includegraphics[height=3in]{images/robert_xy}
        \caption{$I_{xy}$}
    \end{subfigure}%
    ~
    \begin{subfigure}[t]{0.5\textwidth}
        \centering
        \includegraphics[height=3in]{images/robert_edge}
        \caption{Edge map}
    \end{subfigure}
    \caption{Robert operator}
\end{figure*}

\paragraph*{Prewitt filter} is a discrete differentiation operator, which behaves similar to the Sobel operator by computing the gradient for the image intensity function as well.
It makes use of the maximum directional gradient. Though it is easy to implement, it is very sensitive to noise as well.

\begin{equation}
\label{eq:robert_op}
\begin{cases}
	G_x = \begin{bmatrix}
		-1 & 0 & 1 \\ -1 & 0 & 1 \\ -1 & 0 & 1 
	\end{bmatrix} \ast I\\
	G_y = \begin{bmatrix}
		-1 & -1 & -1 \\ 0 & 0 & 0 \\ 1 & 1 & 1
	\end{bmatrix} \ast I
\end{cases}
\end{equation}
The final gradient result calculates through \autoref{eq:gx_gy_gxy} and \autoref{eq:edge_map} as well.

\begin{figure*}[ht!]
    \centering
    \begin{subfigure}[t]{0.5\textwidth}
        \centering
        \includegraphics[height=3in]{images/prewitt_x}
        \caption{$I_x$}
    \end{subfigure}%
    ~ 
    \begin{subfigure}[t]{0.5\textwidth}
        \centering
        \includegraphics[height=3in]{images/prewitt_y}
        \caption{$I_y$}
    \end{subfigure}
    ~
    \vskip\baselineskip
    \begin{subfigure}[t]{0.5\textwidth}
        \centering
        \includegraphics[height=3in]{images/prewitt_xy}
        \caption{$I_{xy}$}
    \end{subfigure}%
    ~
    \begin{subfigure}[t]{0.5\textwidth}
        \centering
        \includegraphics[height=3in]{images/prewitt_edge}
        \caption{Edge map}
    \end{subfigure}
    \caption{Prewitt operator}
\end{figure*}

Noted that since Prewitt kernel can be decomposed, $G_x$ can be written as 
\begin{equation}
	G_x = \begin{bmatrix}
		-1 & 0 & 1 \\ -1 & 0 & 1 \\ -1 & 0 & 1 
	\end{bmatrix} \ast I
	= \begin{bmatrix}
		1 \\ 1 \\ 1 
	\end{bmatrix} \begin{bmatrix}
 		-1 & 0 & 1	
	\end{bmatrix}
	\ast I
\end{equation}
which essentially means that it computes the gradient with smoothing, since it can be decomposed as the products of an averaging and a differential kernel.

\paragraph*{Sobel filter} detects edges where the gradient magnitude is high. 
This makes the Sobel edge detector more sensitive to diagonal edge than horizontal (\autoref{fig:sobel_x}) and vertical edges (\autoref{fig:sobel_y}). 
Both Sobel filter and Prewitt filter are very effective at providing good edge maps.

\begin{equation}
\begin{cases}
	G_x = \begin{bmatrix}
		1 & 0 & -1 \\ 2 & 0 & -2 \\ 1 & 0 & -1 
	\end{bmatrix} \ast I \\
	G_y = \begin{bmatrix}
		1 & 2 & 1 \\ 0 & 0 & 0 \\ -1 & -2 & -1
	\end{bmatrix} \ast I
\end{cases}
\end{equation}

\noindent
The final gradient result calculates through \autoref{eq:gx_gy_gxy} and \autoref{eq:edge_map} as well.


\begin{figure*}[ht!]
    \centering
    \begin{subfigure}[t]{0.5\textwidth}
        \centering
        \includegraphics[height=3in]{images/sobel_x}
        \caption{$I_x$}
        \label{fig:sobel_x}
    \end{subfigure}%
    ~ 
    \begin{subfigure}[t]{0.5\textwidth}
        \centering
        \includegraphics[height=3in]{images/sobel_y}
        \caption{$I_y$}
        \label{fig:sobel_y}
    \end{subfigure}
    ~
    \vskip\baselineskip
    \begin{subfigure}[t]{0.5\textwidth}
        \centering
        \includegraphics[height=3in]{images/sobel_xy}
        \caption{$I_{xy}$}
    \end{subfigure}%
    ~
    \begin{subfigure}[t]{0.5\textwidth}
        \centering
        \includegraphics[height=3in]{images/sobel_edge}
        \caption{Edge map}
    \end{subfigure}
    \caption{Sobel operator}
\end{figure*}

\subsection*{Second-order edge detection}
\paragraph*{Laplacian of Gaussian}
Laplacian filters are derivative filters used to find areas of rapid changing edges in images.

\begin{equation}
	\nabla^2 f(x, y) = \frac{\partial^2 f(x, y)}{\partial x^2} + \frac{\partial^2 f(x, y)}{\partial y^2}
\end{equation}

Since derivative filters are very sensitive to noise, it is common to smooth the image using a Gaussian filter before applying the Laplacian.

A possible approximation for the effect of the Laplacian is 
\begin{equation}
	K_L = \begin{bmatrix}
		0 & 1 & 0 \\ 1 & -4 & 1 \\ 0 & 1 & 0
	\end{bmatrix}
\end{equation}
One can reverse the sign of the elements of this negative Laplacian, but it does not affect the outcome.

To include a smoothing Gaussian lowpass filter, combine the Laplacian and Gaussian functions to obtain the result.
\begin{equation}
	K_{LoG} = -\frac{1}{\pi \sigma^4} \lbrack 1-\frac{x^2+y^2}{2 \sigma^2} \rbrack e^{-\frac{x^2 + y^2}{2 \sigma^2}}
\end{equation}
To utilize functions from previous homework, the calculation is still split instead of using a single kernel.

\begin{figure*}[ht!]
    \centering
    \begin{subfigure}[t]{0.5\textwidth}
        \centering
        \includegraphics[height=3in]{images/log_raw}
        \caption{$I_1$}
    \end{subfigure}%
    ~ 
    \begin{subfigure}[t]{0.5\textwidth}
        \centering
        \includegraphics[height=3in]{images/log_g}
        \caption{$LPF(I_1)$}
        \label{fig:log_g}
    \end{subfigure}
    ~
    \vskip\baselineskip
    \begin{subfigure}[t]{0.5\textwidth}
        \centering
        \includegraphics[height=3in]{images/log_log}
        \caption{$LoG(I_1)$}
        \label{fig:log_log}
    \end{subfigure}%
    ~
    \begin{subfigure}[t]{0.5\textwidth}
        \centering
        \includegraphics[height=3in]{images/log_edge}
        \caption{Edge map}
    \end{subfigure}
    \caption{LoG}
\end{figure*}
Both \autoref{fig:log_g} and \autoref{fig:log_log} uses kernel size of 5 and $\sigma=1$. \autoref{fig:log_log} is ranged from $\lbrack -1, 1 \rbrack$. 

\paragraph*{Difference of Gaussians}

\subsection*{Canny edge detection}
\paragraph*{Step 1: Noise reduction}

\paragraph*{Step 2: Compute gradient magnitude and orientation}

\paragraph*{Step 3: Non-maximal suppression}

\paragraph*{Step 4: Hysteretic thresholding}

\paragraph*{Step 5: Connected component labeling}

\subsection*{Edges of noisy image}

\section*{Problem 2}
\subsection*{Edge crisping}

\subsection*{Warping}

\section*{Bonus}

\end{document}
              